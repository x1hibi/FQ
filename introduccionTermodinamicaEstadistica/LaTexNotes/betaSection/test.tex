
%%%%% START PREAMBLE HEADER %%%%%

%%% START REQUIRED PACKAGES %%%
\documentclass[]{report}
\usepackage{xcolor} 
\usepackage[a4paper, total={6in, 10in}]{geometry} 
\usepackage[document]{ragged2e}
%%% END REQUIRED PACKAGES %%%                

%%% START NEW COMMANDS (shortcut) %%%
% This is a paragraph with normal font
\newcommand{\np}[1]{\paragraph{\normalfont{\large{#1}}}}
% This is a text with a color
\newcommand{\ct}[2]{\textcolor{#1}{#2}}
% This is a bold text 
\newcommand{\bt}[1]{\textbf{#1}}
% This is a italic text 
\newcommand{\et}[1]{\emph{#1}}
% This is a underline text 
\newcommand{\ut}[1]{\underline{#1}}
% This is a numbered equation shortcut
\newcommand{\nec}[1]{\begin{equation} #1 \newline \end{equation}}
% This is a equation shortcut
\newcommand{\ec}[1]{\begin{center} $#1$ \end{center}}
% This is a newline shortcut
\newcommand{\n}{\newline}
%%% END NEW COMMANDS (shortcuts) %%%

%%%%% END PREAMBLE HEADER %%%%%

%%%%%%%%%%%%%%%% START DOCUMENT %%%%%%%%%%%%%%%%
\begin{document}


\part*{Summary Introduction Statistical Thermodynamics}


%%%%%%%%%%%%%%%%%%%%%
% START A CHAPTER 1 %
%%%%%%%%%%%%%%%%%%%%%
\chapter*{Statistics}

\justifying

\section*{\LARGE{Class \#1 } \large{22/09/20}}
\subsubsection*{\Large{Concepts}}

\np{\bt{Random variable:} Is a variable whose value depends of a aleatory event,(Eg. throw a dice or a coin.)} 

\np{Exist two class of random variables \bt{discrete} and \bt{continuous} :}

\begin{itemize}
    \item \bt{Discrete:} Are an array of finite values.
    \item \bt{Continuous:} Are an array of infinite values.
\end{itemize}

\paragraph{Discrete variables}

\np{Throw a dice is an example of aleatory event, and the array of discrete values is [1,2,3,4,5,6], if we want to know which are the probability to get the value 5 when throw the dice once, by our logic know that the probability is $\frac{1}{6}$, and this is true, but the reason behind of this is explain}

\np{So we can call $S$ to the array of the discrete values (\et{communly called "sample space"}), and $n$ the number of possibles values that $S$ can be. \\}

\nec{S=[s_{i},s_{i+1},s_{i+2}...s_{n}]}

\np{The same way we can define the array of probabilities for each values.\\}

\nec{P=[p_{i},p_{i+1},p_{i+2}...p_{n}]}

\np{\et{Example}: for a coin $S=[sideA,sideB]$ by logic we know that the probability to get the $sideA$ or $sideB$ must be $\frac{1}{2}$ this is because have two available values for the coin, we normalization the probability to 1, like this: \\}

\nec{\sum_{i=1}^{n}p_{i} = 1}

\np{For a coin the probability must be 1 this is }

\ec{p_{coin}=p_{1}+p_{2} = \frac{1}{2} + \frac{1}{2}= 1}

\np{This tell us and explaint why the probability for each event is $\frac{1}{2}$ and in the dice case is analogous.}

\ec{p_{dice}=p_{1}+p_{2}+ ... +p_{6}  = \frac{1}{6} + \frac{1}{6}+ ... + \frac{1}{6} = 1}

\paragraph{Permutations}

\np{This are the number of possibilities for a array of }

%%%%%%%%%%%%%%%%%%%
% END A CHAPTER 1 %
%%%%%%%%%%%%%%%%%%%


%%%%%%%%%%%%%%%%%%%%%
% START A CHAPTER #X %
%%%%%%%%%%%%%%%%%%%%%
\chapter*{NAME OF CHAPTER}

\section*{\LARGE{Class \#NUMBER OF CLASS } \large{DATE DD/MM/YY}}
\subsubsection*{\Large{Topic X}}

\np{Lorem Ipsum has been the industry's standard dummy text ever since the 1500s, when an unknown printer took a galley of type and scrambled it to make a type specimen book. It has survived not only five centuries, but also
now you can seee \ct{yellow}{text with colot} \bt{bold text} \et{italic text } \ut{underline text}}
%%%%%%%%%%%%%%%%%%%
% END A CHAPTER #X %
%%%%%%%%%%%%%%%%%%%



\end{document}
%%%%%%%%%%%%%%%% END DOCUMENT %%%%%%%%%%%%%%%%
