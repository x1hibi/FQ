

\section*{Objetivos Específicos}

\np{Especificar la metodología empleada para la obtención apropiada de las muestras de epidermis de los especímenes.}

\np{Identificar la morfología normal de las células dendríticas epidérmicas en las muestras procesadas con la histoquímica enzimática para ATPasa para su posterior identificación en las muestras procesadas con las impregnaciones ZIO y cloruro de oro.}

\np{Determinar las condiciones de temperatura, pH, tiempo de reacción y concentración de los reactivos para la obtención de resultados óptimos y repetibles en las impregnaciones ZIO y cloruro de oro.}

\np{Interpretar las morfologías celulares observadas en las muestras procesadas con las impregnaciones para la identificación de las células dendríticas epidérmicas.}

\np{Determinar por medio de conteos en microfotografías el número de células dendríticas que hay en 3mm2 en las muestras procesadas * para la obtención de conjuntos de datos que permitan la comparación estadística de la densidad celular entre las impregnaciones realizadas y la histoquímica enzimática para ATPasa}

\np{Comparar las densidades célulares de las impregnaciones realizadas y la histoquímica enzimática para ATPasa mediante una prueba t de Student para la corroboración de la hipótesis nula planteada.}