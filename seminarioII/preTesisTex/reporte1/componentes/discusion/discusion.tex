
\vspace{-0.5cm}
\section*{Discusión}

La histoquímica enzimática para ATPasa nos muestra la distribución de las CL en epidermis de piel(Figura 4), la cual es irregular, donde se pueden apreciar zonas con una mayor concentración de CL (cúmulos de CL) y otras donde se apreció un arreglo más ordenado y homogéneo.

Comparando la Figura 3 y 4 se contrastan dos escenarios de interés, el primero es la muestra control, que nos muestra como es la epidermis en condiciones normales, donde no es posible identificar a las CL sin una tinción específica para CL y el segundo que es la experimental, la cual nos muestra la especificidad que debe tener una técnica para una correcta identificación, ya que el producto de reacción presenta afinidad hacia las CL y no a otros componentes celulares.

Al observar las CL a mayor aumento (Figuras 5 y 6) se lograron identificar las características morfológicas típicas de las CL, de las cuales destacan su soma alargado e irregular y sus múltiples y largas prolongaciones que forman redes entre ellas.

Con la información recabada de la histoquímica enzimática para ATPasa se podrá llevar a cabo la interpretación correcta de las muestras procesadas con las impregnaciones ZIO y cloruro de oro.

