\section*{Materiales y Métodos}

\subSec{Obtención de muestras}

Se seleccionaron tres especies de vertebrados no mamíferos: pollo \et{Gallus gallus}, rana \et{Lithobates} y pez \et{Bagre marinus}.

Se realizó el sacrificio de los especímenes \cite{web:NOM} y la obtención de muestras de piel de los especímenes, el tejido fue fijado en una solución de formol-cacodilatos durante 24 horas.

Las muestras fueron sumergidas en una solución de $CaCl_{2}$ durante 15 min a $37^oC$ \cite{article:epidermis}, con el microscopio estereoscópico se realizó la separación de la epidermis mediante el uso agujas de disección, las muestras de epidermis fueron lavadas y conservadas en solución salina balanceada (SSB).

\subSec{Histoquímica enzimática para ATPasa}

Las muestras fueron lavadas tres veces en SSB, después fueron transferidas a una solución de buffer de TRIS maleato con pH 7.4 durante 15 min, posteriormente a la solución de $ATP$ (1) y (2) \cite{article:ATPasaReaction} y $ATP_{control}$ respectivamente, estuvieron 1 hora a $37^oC$, las muestras se revelaron con una solución al $1\%\frac{v}{v}$ de $(NH_{4})_{2}S$ al $22\%\frac{m}{v}$, las muestras de esta técnica fueron procesadas en menos de una semana por el tiempo de vida de la enzima ATPasa.

\ec{Mg + ATP \rightleftharpoons Mg-ATP \xrightarrow{\text{ATPasa}} Mg-ADP + P_{i}}

\ec{P_{i}+Pb \rightarrow PbP_{i}\downarrow}

\ec{Pb_3(PO_4)_2 + 3(NH_4)_2S \rightarrow 3PbS\downarrow + 2(NH_4)_3PO_4}



\subSec{Impregnación Cloruro de oro}

Las muestras se pasan a una solución buffer de citratos a pH 3-3.5 por 10 min, se transfieren a una solución de $AuCl_{3}\cdot 3H_{2}O$ a concentraciones de $0.25,0.5,0.75$ y $1\% \frac{m}{v}$ a tiempos de 5, 10, 20, 30 min, 1 y 2 horas, estas pruebas fueron realizadas a temperaturas de $4,~25,~30$ y $35^oC$.

Después se procedió la reducción del $HAuCl_{4}$ (3) formado en medio ácido \cite{article:reduccionOro}, las muestras se transfirieron a una solución de $HCOOH$ al $20,25$ y $30\%\frac{m}{v}$ los tiempos de impregnación fueron de 30 min, 1, 2, 4, 8, 24 y 48 horas a $4~,~25,~30$ y $35^oC$, estas condiciones se realizaron para la impregnación de ZIO.

\ec{2HAuCl_{4} + 3HCOOH \rightarrow 2Au\downarrow + 8HCl + 3CO_{2}}


\subSec{Impregnación ZIO}

Las muestras son transferidas a un buffer tris con pH de 7.4 durante 10 min, después se pasan a la solución de ZIO, la cual es preparada al momento mezclando 4 volúmenes una solución de $ZnI_{2}$ al $3\%\frac{m}{v}$ con uno de $OsO_{4}$ al $2\%\frac{m}{v}$, donde el producto de reacción es el $ZnOsO_{4}$ \cite{article:productoZIO}.

\subSec{Identificación de CL}

Las muestras después de su procesamiento fueron montadas con resina sintética y estas fueron observadas en el microscopio óptico, las laminillas de $ATP$ y $ATP_{control}$ fueron comparadas para saber que la reacción enzimática se llevó a cabo, se observó la morfología de las CL en los especímenes seleccionados y posteriormente se realizó interpretación de las laminillas procesadas con las impregnaciones.

\subSec{Análisis estadístico}

Se tomaron microfotografías de zonas aleatorias de las laminillas con el microscopio óptico-digital, de la cual se obtuvo un área de $3mm^{2}$ de cada técnica realizada y se realizó el conteo de CL en cada imagen, con los datos obtenidos se realizó la comparación de la densidad de CL empleando la prueba estadística t de Student.

\subSec{Costos}

Se usaron los precios de Sigma-Aldrich para los siguientes reactivos ATP(5g 2,926 pesos), $HAuCl_{4}$ (1g 4,742 pesos) y $OsO_{4}$ (1g 6,952 pesos), y el microscopio óptico-digital (1 unidad \~20,000 pesos), considerando el tiempo de uso del microscopio de 24 a 48 horas se considera un aproximado de 500 pesos, por lo que se cotiza un total 15,120 pesos.

\subSec{Tratamiento de residuos}

Los residuos de la solución de ATP, $(NH_{4})S_{2}$, osmio y cloruro de oro fueron almacenados y rotulados en frascos ámbar para su posterior tratamiento por la Facultad de Química.


