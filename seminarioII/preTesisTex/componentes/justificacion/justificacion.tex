

\section*{Justificación}

\np{Las técnicas disponibles para el estudio histológico de las células dendríticas epidérmicas(CDE) presentan limitaciones de disponibilidad de anticuerpos para diversas especies, así como elevados costos(inmunohistoquímica), tiempos reducidos para el procesamiento de las muestras biológicas histoquímica enzimática para ATPasa) y el limitado acceso a instrumental de alto costo(microscopio electrónico de transmisión); El demostrar que las impregnaciones ZIO y cloruro de oro son capaces de la identificación específica de las CDE permitirá el desarrollo de estudios histológico obteniendo información mediante análisis cuantitativo y morfométrico de las CDE en vertebrados no mamíferos, lo cual es de gran interés debido a que las características que están presentes en dicho grupo no están presentes en los mamíferos, además como ampliar el conocimiento y el entendimiento del sistema inmune, lo cual tiene impacto en el área de la salud ya que puede contribuir al desarrollo de tratamientos de enfermedades relacionadas con las CDE como melanoma, vitíligo, Leishmaniasis y histiocitosis X.}