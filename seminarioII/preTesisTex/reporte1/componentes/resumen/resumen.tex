%%% THIS CONTENT IS IN ONE COLUMN (START) %%%
\twocolumn[
    \begin{@twocolumnfalse}

        %% CREATE A TITLE (START) %%
        \maketitle
        %% CREATE A TITLE (END) %%

        \selectlanguage{spanish} 
        %% CREATE A ABSTRACT (START,MAX 250 CHARACTERS) %%
        \begin{abstract}
            \item El estudio de las células de langerhans en vertebrados no mamíferos es reducido y una de las causas es la falta de técnicas capaces de identificarlas de forma específica en la mayoría de las especies de vertebrados, el analizar a las impregnaciones ZIO y cloruro de oro como técnicas generales de identificación de células de langerhans, lo cual proveerá herramientas para su estudio en los vertebrados no mamíferos, se realizaron impregnaciones  ZIO, cloruro de oro y la histoquímica enzimática para ATPasa en las especies pollo \et{Gallus gallus}, rana \et{Lithobates} y pez \et{Bagre marinus} con diferentes condiciones de pH, temperatura, tiempo de reacción y concentraciones de reactivos, se comparó la morfología obtenida entre las impregnaciones y la histoquímica como técnica de referencia, así una comparación estadística de la densidad celular por medio de una prueba T Student, por medio de la histoquímica se observó la morfología y distribución normal de las células de Langerhans en los especímenes de estudio.
            \item \bt{Palabras clave:} \et{célula de Langerhans}, \et{impregnación ZIO}, \et{impregnación de cloruro de oro}, \et{histoquímica enzimática para ATPasa}, \et{vertebrados no mamíferos}. 
        \end{abstract}
        
        %% CREATE A ABSTRACT (END) %%

    \end{@twocolumnfalse}
]
%%% THIS CONTENT IS IN ONE COLUMN (END) %%%